
\documentclass[]{article}
\usepackage{tikz}
\usepackage{tikz,fullpage}
\usetikzlibrary{arrows,%
	petri,%
	topaths}%
\usepackage{tkz-berge}
\usepackage[position=top]{subfig}
\usepackage{amsmath}
\usepackage{amsfonts}
\usepackage{amssymb}
\usepackage{graphicx}
\usepackage{textcomp}
\usepackage{logicproof}
\usepackage{tabularx}
\usepackage{float}
\usepackage[linesnumbered,ruled]{algorithm2e}
%opening

\title{Algorithms \& Data Structures 2018/19 Coursework}
\author{clvp22}


\begin{document}
\maketitle

\section*{Question 4.}
\subsection*{(a)}
Firstly we can say $x^3+3x +2$ is $O(x^3)$ because there exists some $k$ and $C$ such that,
$$x^3+3x+2 \leq C \cdot x^3 $$
for every $x \geq k$.
\\
(example; $C=6$, $k=0$)
\\
\\
So, $x^3+3x +2$ is $O(x^3)$. And $2x^4$ is $O(x^3+3x +2)$ if and only if $2x^4$ is $O(x^3)$. We now can assume there are some constants $k$ and $C$ such that,
$$2x^4 \leq C \cdot x^3$$
for every $x \geq k$.
\\
\\
So,
$$ C \geq 2x$$
however, $2x$ grows monotonically for $x > 0$. This is a contradiction because then no constant value for $C$ can be found for $k>0$. Therefore $2x^4$ is not $O(x^3)$ and consequently not $O(x^3+3x +2)$.
\subsection*{(b)}
For $x \geq 2$, $1 \leq \log x \leq x \leq x^2 \leq x^3$.
\\
\\
So,
$$4x^3+2x^2 \cdot \log x +1 \leq 4x^3+2x^2 \cdot x + x^3 = 8x^3$$
for every $x \geq k$.
\\
\\
This inequality holds using $k=2$ and $C=8$ as a pair of witnesses,
$$4x^3 +2x^2 \cdot \log x +1 \leq C \cdot x^3 $$
for every $x \geq k$.
\\
\\
Therefore $4x^3+2x^2 \cdot \log x +1$ is $O(x^3)$.
\subsection*{(c)}
$3x^2+7x+1$ is $\omega (x \cdot \log x)$ if and only if $x \cdot \log x$ is $o(3x^2+7x+1)$, which is,
$$\displaystyle{\lim_{x \to \infty}} \frac{C \cdot x \log x}{3x^2+7x+1}$$
for every $x \geq k$.
\\
\\
For $x \geq 2$, $1 \leq \log x \leq x \leq x^2$. 
\\
So $3x^2+7x+1$ grows faster than $x \cdot \log x$ so $\displaystyle{\lim_{x \to \infty}} \frac{C \cdot x \log x}{3x^2+7x+1}$ holds for $k = 2$ and $C=1$. Therefore, $x \cdot \log x$ is $o(3x^2+7x+1)$, and consequently $3x^2+7x+1$ is $\omega (x \cdot \log x)$.
\subsection*{(d)}
For $x \geq 2$, $\log x \leq x \leq x^2$.
\\
\\
So,
$$x^2+4x \geq x^2 = x \cdot x \geq x \cdot \log x$$
for every $x \geq k$.
\\
\\
This inequality holds using $k = 2$ and $C = 1$ as a pair of witnesses,
$$x^2+4x \geq C \cdot x \log x$$
Therefore $x^2+4x$ is $\Omega (x \cdot \log x)$.
\subsection*{(e)}
$f(x)+g(x)$ is $\Theta (f(x) \cdot g(x))$, if $f(x)+g(x)$ is $O(f(x) \cdot g(x))$ and if $f(x)+g(x)$ is $\Omega (f(x) \cdot g(x))$.
\\
Suppose $f(x)$ is $O(g(x))$, or in other words, 
$$ f(x) \leq C \cdot g(x)$$
We can now ignore $f(x)$ from our previous statement as $g(x)$ dominates $f(x)$. Our question can be re-constructed as follows,
\\
\\
Is $g(x)$ $\Theta (f(x) \cdot g(x))$, for some $f(x)$, where $f(x)$ is $O(g(x))$.
\\
\\
Again, $g(x)$ is $\Theta (f(x) \cdot g(x))$, if $g(x)$ is $O(f(x) \cdot g(x))$ and if $g(x)$ is $\Omega (f(x) \cdot g(x))$. 
\\
Or in other words,
$$g(x) \leq C \cdot f(x) \cdot g(x) $$
and
$$g(x) \geq C \cdot f(x) \cdot g(x)$$
so,
$$C_{1} \cdot f(x) \cdot g(x) \leq g(x) \leq C_{2} \cdot f(x) \cdot g(x) $$
Clearly this inequality can only be satisified if and only if $f(x)$ is a non-zero constant function. If however we said $g(x)$ is $O(f(x))$ then the same principles can be applied and the original statement could only be satisified if and only if $g(x)$ is a non-zero constant funtcion instead. So $f(x)+g(x)$ is not $\Theta (f(x) \cdot g(x))$ for non-constant functions $f(x)$ and $g(x)$.
\end{document}
